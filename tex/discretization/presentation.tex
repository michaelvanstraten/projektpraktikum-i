\documentclass[9pt, t]{beamer}
\usepackage[ngerman]{babel} 
\usepackage{amsmath, amssymb, mathtools}
\usepackage{subcaption}
\usepackage[
backend=biber,
style=alphabetic,
sorting=ynt
]{biblatex}
\bibliography{presentation.bib}

\graphicspath{{./figures/}}

\usetheme{Berlin}

% Disable navigation buttons
\setbeamertemplate{navigation symbols}{}
\setbeamercovered{transparent}
\newcommand{\margin}{0.05\paperwidth}
\beamersetrightmargin{\margin}
\beamersetleftmargin{\margin}

\newcommand\N{\mathbb{N}}
\newcommand\R{\mathbb{R}}
\newcommand\BigO{\mathcal{O}}

\DeclarePairedDelimiterXPP{\set}[1]{}{\lbrace}{\rbrace}{}{#1}
\DeclarePairedDelimiterXPP{\paren}[1]{}{\lparen}{\rparen}{}{#1}

\title{Poisson-Problem}
\author{P. Merz und M. van Straten}
\institute{Humboldt-Universität zu Berlin \\
           Wintersemester 2024}
\date{\today}

\begin{document}

\maketitle

\begin{frame}{Inhalt}
    \tableofcontents[pausesections]
\end{frame}

\section{Einführung}

\begin{frame}{Einleitung und Motivation}
    Die Poisson-Gleichung ist eine elliptische partielle Differenzialgleichung, wichtig in:
    \begin{itemize}[<+->]
        \item Gravitationstheorie,
        \item Elektrostatik,
        \item Strömungsmechanik.
    \end{itemize}
    \pause%
    Aufgrund der Schwierigkeit analytischer Lösungen wird eine numerische
    Approximation untersucht\ \cite{Poisson}.
\end{frame}

\section{Theorie}

\subsection{Poisson-Problem}

\begin{frame}{Poisson-Problem}
    Gesucht ist \(u \in C^2(\R^2; \R)\), das die
    Poisson-Gleichung auf einem Gebiet \(\Omega \subset \R^2\) erfüllt:
    \pause%
    \begin{align*}
        -\Delta u & = f \quad \text{in} \quad \Omega,           \\
        u         & = g \quad \text{auf} \quad \partial \Omega.
    \end{align*}
    \pause%
    Dabei ist
    \[
        \Delta u = \frac{\partial^2 u}{\partial x_1^2} + \frac{\partial^2 u}{\partial x_2^2}
    \]
    der Laplace-Operator.\ \pause%
    \begin{block}{Spezielle Annahmen für Experimente}
        \begin{itemize}[<+->]
            \item Gebiet: \(\Omega = {(0, 1)}^2\)
            \item Randbedingung: \(g = 0\)
        \end{itemize}
    \end{block}
\end{frame}

\subsection{Diskretisierung des Gebietes}

\begin{frame}{Diskretisierung des Gebietes}
    \begin{itemize}[<+->]
        \item Das Intervall \((0,1)\) wird in \(n\) gleich lange Teile
              unterteilt:
              \[
                  X_1 = \set*{ \frac{j}{n} \mid 1 \leq j \leq n-1}.
              \]
        \item Daraus wird ein Gitter aus äquidistanten Punkten konstruiert:
              \[
                  X
                  = X_1 \times X_1
                  = \set*{\paren*{\frac{j}{n}, \frac{k}{n}} \mid 1 \leq j, k \leq n-1}.
              \]
        \item Das Gitter umfasst \(N \coloneq {(n - 1)}^2\) Punkte.
    \end{itemize}
\end{frame}

\subsection{Diskretisierung des Laplace-Operators}

\begin{frame}{Diskretisierung des Laplace-Operators}
    Mithilfe der finiten Differenzen Methode wird der Laplace-Operator approximiert:
    \pause%
    \[
        \frac{\partial^2 u}{\partial x_1^2} (v, w)
        \approx \frac{u(v + h, w) - 2u(v, w) + u(v - h, w)}{h^2}
    \]
    \[
        \frac{\partial^2 u}{\partial x_2^2} (v, w)
        \approx \frac{u(v, w + h) - 2u(v, w) + u(v, w - h)}{h^2}
    \]
    \pause%
    \[
        \Delta u
        \approx \frac{u(v + h, w) + u(v, w + h)
            - 4u(v, w) + u(v - h, w) + u(v, w - h)}{h^2}
    \]
    \pause%
    \begin{itemize}[<+->]
        \item \(h = \frac{1}{n}\)
        \item Der diskretisierte Laplace-Operator wird als \(\Delta_h u\)
              bezeichnet \ \cite{PPI_Poisson}.
    \end{itemize}
\end{frame}

\subsection{Aufstellen des Gleichungssystems}

\begin{frame}{Problemformulierung}
    Gesucht ist eine Lösung für:
    \begin{align*}
        -\Delta_h u(x) & = f \quad \text{für} \quad x \in X,         \\
        u(x)           & = 0 \quad \text{auf} \quad \partial \Omega.
    \end{align*}
    \pause%

    Dieses Problem führt auf ein lineares Gleichungssystem (LGS) mit \(N =
    {(n-1)}^2\) Gleichungen.

    Die Gleichungen werden mittels der Bijektion geordnet:
    \begin{align*}
        \operatorname{idx}: \set{1, \ldots, n - 1}^2 & \longrightarrow \set*{1, \ldots , N}, \\
        (j, k)                                       & \longmapsto (k - 1)(n - 1) + j.
    \end{align*}
\end{frame}

\begin{frame}{Matrixstruktur des Gleichungssystems}
    Die Matrix \(A \in \R^{N \times N}\) hat die Blockstruktur:
    \[
        A \coloneq
        \begin{bmatrix}
            C      & -I     & 0      & \cdots & 0      \\
            -I     & C      & -I     & \cdots & 0      \\
            0      & \ddots & \ddots & \ddots & \vdots \\
            \vdots & \ddots & \ddots & \ddots & -I     \\
            0      & \cdots & 0      & -I     & C
        \end{bmatrix}.
    \]
    \pause%
    Hierbei ist \(I\) die Einheitsmatrix.
\end{frame}

\begin{frame}{Submatrix \(C\)}
    Die Blockmatrix \(C \in \R^{(n-1) \times (n-1)}\) hat die Struktur:
    \[
        C \coloneq
        \begin{bmatrix}
            4      & -1     & 0      & \cdots & 0      \\
            -1     & 4      & -1     & \cdots & 0      \\
            0      & \ddots & \ddots & \ddots & \vdots \\
            \vdots & \ddots & \ddots & \ddots & -1     \\
            0      & \cdots & 0      & -1     & 4
        \end{bmatrix}.
    \]
    \pause%
    Diese Struktur ergibt sich aus der Diskretisierung des Laplace-Operators.
\end{frame}

\subsection{LU-Zerlegung}

\begin{frame}{LU-Zerlegung}
    \begin{block}{Definition}
        Für \(A \in \R^{n \times n}\) existieren:
        \[
            A = PLU,
        \]
        wobei \(P\) eine Permutationsmatrix, \(L\) eine untere Dreiecksmatrix
        und \(U\) eine obere Dreiecksmatrix ist~\cite{LU}.
    \end{block}
    \pause%
    \begin{block}{Lösung eines LGS}
        \[
            A x = b \quad \Rightarrow \quad LU x = P^T b
        \]
    \end{block}
    \pause%
    \begin{block}{Komplexität}
        \begin{itemize}[<+->]
            \item Zerlegung: \(\BigO(n^3)\)
            \item Lösung: \(\BigO(n^2)\)
        \end{itemize}
    \end{block}
\end{frame}

\section{Experimente}

\begin{frame}{Rahmenbedingungen}
    \begin{block}{Analytische Lösung}
        \[
            u(x) \coloneq x_1 \cdot \sin(3 \cdot \pi \cdot x_1) \cdot x_2 \cdot \sin(3 \cdot \pi \cdot x_2)
        \]
    \end{block}
    \pause%
    \begin{block}{Numerische Approximation}
        Numerische Lösung \(\hat u\) wurde für das Poisson-Problem unter
        Einbeziehung der Anfangsbedingungen berechnet und mit der analytischen Lösung \(u\) verglichen.
    \end{block}
    \pause%
    \begin{block}{Zentrale Fragestellungen}
        \begin{enumerate}[<+->]
            \item Optimales Speicherformat der Systemmatrix \(A\).
            \item Konvergenzverhalten der numerischen Approximation.
        \end{enumerate}
    \end{block}
\end{frame}

\subsection{Sparsity}

\begin{frame}{Experimente: Untersuchung der Sparsity}
    \begin{block}{Zielsetzung}
        Analyse der Dünnbesetztheit (Sparsity) der Systemmatrix \(A\) sowie ihrer LU-Zerlegung.
    \end{block}
    \pause%
    \begin{block}{Untersuchung}
        \begin{itemize}[<+->]
            \item Struktur von \(A\) in Abhängigkeit von \(N\).
            \item Dünnbesetztheit der LU-Zerlegung von \(A\).
        \end{itemize}
    \end{block}
\end{frame}

\begin{frame}{Sparsity der Matrix \(A\)}
    \centering
    \includegraphics[width=\textwidth]{sparsity.pdf}
\end{frame}

\begin{frame}{Sparsity der LU-Zerlegung von \(A\)}
    \centering
    \includegraphics[width=\textwidth]{sparsity-lu.pdf}
\end{frame}

% \begin{frame}{Beobachtungen und Auswertung}
%     \begin{itemize}[<+->]
%         \item Gesamteinträge in \(A\) skalieren proportional zu \(N^2\)
%               (Log-Log-Darstellung).
%         \item Nicht-Nulleinträge wachsen proportional zu \(N\), was langsamer
%               ist.
%         \item Relative Sparsity: Abfall proportional zu \(N^{-1}\).
%         \item Nicht-Nulleinträge der LU-Zerlegung wachsen proportional zu
%               \(n^3\).
%         \item Relative Sparsity der LU-Zerlegung: Abfall proportional zu
%               \(n^{-1}\).
%     \end{itemize}
% \end{frame}
%
\subsection{Vergleich der analytischen und numerischen Lösungen}

\begin{frame}{Untersuchung: Numerische vs. Analytische Lösung}
    \begin{block}{Zielsetzung}
        Vergleich der numerischen Lösung \(\hat u\) mit der analytischen Lösung \(u\)
        für unterschiedliche Gittergrößen \(n \in \{4, 11, 128\}\).
    \end{block}
    \pause%
    \begin{block}{Visualisierung}
        Grafische Darstellung der Lösungen zur Beurteilung der Annäherung.
    \end{block}
\end{frame}

\begin{frame}{Lösungsvergleich für \(n = 4\)}
    \centering
    \includegraphics[width=\textwidth]{solutions-for-n-equal-4.pdf}
\end{frame}

\begin{frame}{Lösungsvergleich für \(n = 11\)}
    \centering
    \includegraphics[width=\textwidth]{solutions-for-n-equal-11.pdf}
\end{frame}

\begin{frame}{Lösungsvergleich für \(n = 128\)}
    \centering
    \includegraphics[width=\textwidth]{solutions-for-n-equal-128.pdf}
\end{frame}

% \begin{frame}{Beobachtungen und Auswertung}
%     \begin{itemize}[<+->]
%         \item Mit steigendem \(n\) nähert sich die numerische Lösung \(\hat u\)
%               der analytischen Lösung \(u\) immer besser an.
%         \item Grafische Vergleiche zeigen eine deutliche Verbesserung der
%               Übereinstimmung bei höheren Gitterauflösungen.
%     \end{itemize}
% \end{frame}
%
\subsection{Fehleranalyse}

\begin{frame}{Untersuchung des Fehlers}
    \begin{block}{Zielsetzung}
        Systematische Analyse des Fehlers zwischen der approximierten Lösung \(\hat u\) und der exakten Lösung \(u\).
    \end{block}
    \pause%
    \begin{block}{Ansätze}
        \begin{itemize}[<+->]
            \item Darstellung der Fehlerverteilung in 3D und als Heatmap für
                  \(n \in \{4, 11, 128\}\).
            \item Analyse des maximalen Fehlers in Abhängigkeit von \(N\).
        \end{itemize}
    \end{block}
\end{frame}

\begin{frame}{Fehlerverteilung für \(n = 4\)}
    \centering
    \includegraphics[width=\textwidth]{difference-for-n-equal-4.pdf}
\end{frame}

\begin{frame}{Fehlerverteilung für \(n = 11\)}
    \centering
    \includegraphics[width=\textwidth]{difference-for-n-equal-11.pdf}
\end{frame}

\begin{frame}{Fehlerverteilung für \(n = 128\)}
    \centering
    \includegraphics[width=\textwidth]{difference-for-n-equal-128.pdf}
\end{frame}

\begin{frame}{Maximaler Fehler in Abhängigkeit von \(n\)}
    \centering
    \includegraphics[width=0.9\textwidth]{error.pdf}
\end{frame}

% \begin{frame}{Beobachtungen und Auswertung}
%     \begin{itemize}[<+->]
%         \item Der maximale Fehler fällt zunächst, steigt kurz an und fällt
%               schließlich proportional zu \(N^{-1}\) (Log-Log-Darstellung).
%         \item Für größere Werte von \(n\) wird der Fehler insgesamt kleiner (z.
%               B. bei \(n = 128\) liegt er zwischen 0.00005 und 0.00030).
%         \item Die numerische Lösung \(\hat u\) konvergiert mit wachsendem \(n\)
%               gegen die analytische Lösung \(u\), da feinere Diskretisierung
%               mehr Punkte des Gebiets \(\Omega\) abdeckt.
%         \item Dieses Verhalten ist konsistent mit der theoretischen Erwartung.
%     \end{itemize}
% \end{frame}
%
\section{Zusammenfassung}

\begin{frame}{Zusammenfassung}
    \textbf{Erkenntnisse aus den Experimenten:}
    \begin{itemize}[<+->]
        \item Theoretischer Speicherplatzbedarf für \(A\) liegt im
              vollbesetzten Format in \(\BigO(n^4)\) und im sparse Format in
              \(\BigO(n^2)\)
        \item Sparsity der LU-Zerlegung in \(\BigO(n^3)\).
    \end{itemize}

    \textbf{Vergleich: Analytische vs numerische Lösung}
    \begin{itemize}[<+->]
        \item Numerische Lösung sieht mit wachsendem \(n\) immer mehr wie die
              die analytische Lösung aus.
    \end{itemize}

    \textbf{Fehleranalyse:}
    \begin{itemize}[<+->]
        \item Maximaler Fehler zwischen analytischer und numerischer Lösung
              liegt in \(\BigO(N^{-1})\)
    \end{itemize}
\end{frame}

\begin{frame}{Literaturverzeichnis}
    \printbibliography%
\end{frame}

\end{document}
