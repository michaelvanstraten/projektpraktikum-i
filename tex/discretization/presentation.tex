\documentclass[9pt, t]{beamer}
\usepackage[ngerman]{babel} 
\usepackage{textgreek}
\usepackage{amsmath}
\usepackage{amssymb}
\usepackage{mathtools}
\usepackage[
backend=biber,
style=alphabetic,
sorting=ynt
]{biblatex}
\bibliography{presentation.bib}
\usepackage{svg}

\usepackage{import}

\usepackage{subcaption}

% Disable navigation buttons
\setbeamertemplate{navigation symbols}{}
\setbeamercovered{transparent}
\newcommand{\margin}{0.05\paperwidth}
\beamersetrightmargin{\margin}
\beamersetleftmargin{\margin}


\title{Poisson-Problem}
\author{M. van Straten und P. Merz}
\institute{Humboldt-Universität zu Berlin \\
           Wintersemester 2024}
\date{\today}

\usetheme{Berlin}

\begin{document}

\maketitle

\begin{frame}{Inhalt}
    \tableofcontents[pausesections]
\end{frame}










\section{Einführung}


\begin{frame}{Einleitung und Motivation}

Die Poisson-Gleichung ist eine elliptische partielle Differenzialgleichung von zentraler Bedeutung in Bereichen wie:

\begin{itemize}
    \item Gravitationstheorie
    \item Elektrostatik
    \item Strömungsmechanik
\end{itemize}

Da analytische Lösungen schwierig zu finden sind, wird eine numerische Approximation zur Lösung des Problems untersucht.
\end{frame}












\section{Theorie}

\begin{frame}{Poisson-Problem}
  Das Poisson-Problem besteht in der Bestimmung einer Funktion \(u \in C^2(\mathbb{R}^2; \mathbb{R})\), die für ein gegebenes Gebiet \(\Omega \subset \mathbb{R}^2\) mit Rand \(\partial \Omega\) sowie zwei vorgegebene Funktionen \(f \in C(\Omega; \mathbb{R})\) und \(g \in C(\partial \Omega; \mathbb{R})\) die folgenden Bedingungen erfüllt:
  \begin{align*} 
    -\Delta u &= f \quad \text{in} \quad \Omega 
    \\ 
            u &= g \quad \text{auf} \quad \partial \Omega
    
  
  \end{align*}
  wobei \(\Delta u = \frac{\partial^2 u}{\partial x_1^2} + \frac{\partial^2 u}{\partial x_2^2}\) der Laplace-Operator von \(u\) ist.
\end{frame}  

\begin{frame}{Diskretisierung des Gebietes}
\begin{itemize}
    \item 
    Zuerst wird das Intervall \((0,1)\) in \(n\) gleich lange Teile unterteilt
    \[
     X_1 = \left\{ \frac{j}{n} \mid 1 \leq j \leq n-1 \right\}
    \]
    \item 
    Hieraus wird ein Gitter an äquidistanten Punkten konstruiert
    \[
        X = X_1 \times X_1 = \left\{ \left( \frac{j}{n}, \frac{k}{n} \right) \mid 1 \leq j, k \leq n-1 \right\}
    \]
    \item 
    Das Gitter besteht somit aus \(N \coloneq(n - 1)^2\) Gitterpunkten
\end{itemize}
\end{frame}

\begin{frame}{Diskretisierung des Laplace-Operators}
Mithilfe der finiten Differenzen Methode gilt:
\[
    \frac{\partial^2 u}{\partial x_1^2} (v, w) \approx \frac{u(v + h, w) - 2u(v, w) + u(v - h, w)}{h^2}
\]
\[
    \frac{\partial^2 u}{\partial x_2^2} (v, w) \approx \frac{u(v, w + h) - 2u(v, w) + u(v, w - h)}{h^2}
\]
Somit folgt
\begin{align*}
    \Delta u & = \frac{\partial^2 u}{\partial x_1^2} + \frac{\partial^2 u}{\partial x_2^2}           \\
             & \approx \frac{u(v + h, w) + u(v, w + h) - 4u(v, w) + u(v - h, w) + u(v, w - h)}{h^2}
    \\
             & \eqcolon \Delta_h u
\end{align*}
Mit \(\Delta_h u\) wird der diskretisierte Laplace-Operator bezeichnet
\end{frame}



\begin{frame}{Aufstellen des Gleichungssystems}
 Gesucht ist eine Lösung für:

\[
    -\Delta_h u(x) = f \quad \text{für} \quad x \in X
\]
\[
    u(x) = 0 \quad \text{auf} \quad \partial \Omega
\]

Dies ist ein lineares Gleichungssystem mit \(N\) Gleichungen.
Geordnet werden die Gleichungen mittels der Bijektion
\begin{align*}
    \operatorname{idx}: \{1, \ldots, n - 1\}^2 & \longrightarrow \{1, \ldots , N\}, \\
    (j, k)                                     & \longmapsto (k - 1)(n - 1) + j.
\end{align*}

%TO DO ADD DESCRIPTION OF MATRIX

\end{frame}


\begin{frame}
\frametitle{LU-Zerlegung}

Sei \(A \in \mathbb{R}^{n \times n}\) eine quadratische Matrix. Dann existieren eine
Permutationsmatrix \(P \in \R^{n \times n}\), eine linke untere Dreiecksmatrix
\(L \in \mathbb{R}^{n \times n}\) und eine rechte obere Dreiecksmatrix \(U \in \mathbb{R}^{n
\times n}\), sodass gilt:
\[
    A = PLU.
\]

Die LU-Zerlegung  hilft, das lineare Gleichungssystem effizient zu lösen:

\[
    A x = b \quad \Rightarrow \quad PLU x = b \quad \Rightarrow \quad LU x = P^T b
\]
\begin{itemize}
    \item Bestimmen der Matrizen \(P, L, U\) liegt in \(\mathcal{O}(n^3)\)
    \item Lösen des Gleichungssystems mittels Vorwärts- und Rückwärtssubstitution liegt in \(\mathcal{O}(n^2)\)
\end{itemize}
\end{frame}

















\section{Experimente}

\begin{frame}{Rahmenbedingungen}
    \begin{itemize}
        \item Untersucht wurde die Funktion \(u(x) \coloneq x_1 \cdot \sin(3 \cdot x_1) \cdot x_2 \cdot \sin(3 \cdot x_2)\)
        \item Es gilt \(f(x) \coloneq -6 \paren*{\begin{array}{l}
            x_1 \cdot \cos(3 \cdot x_2) \cdot \sin(3 \cdot x_1)   \\
            + x_2 \cdot \cos(3 \cdot x_1) \cdot \sin(3 \cdot x_2) \\
            - 3 \cdot x_1 \cdot x_2 \sin(3 \cdot x_1) \cdot \sin(3 \cdot x_2)
        \end{array}}
    = -\Delta u(x). \)
        \item Die Funktion \(f\) wird als Input für das Poisson-Problem verwendet
        und die numerische Lösung mit der analytische Lösung \(u\) verglichen.
    \end{itemize}
    
\end{frame}

\begin{frame}{Vergleich: A als vollbesetzte und dünnbesetzte Matrix}



%TO DO ADD GRAPH


%TO DO ADD BEOBACHTUNG
\end{frame}



\begin{frame}{Graphischer Vergleich: Analytische und numerische Lösung}



%TO DO ADD GRAPH


\textbf{Beobachtung:}
\begin{itemize}
    \item Numerische Lösung sieht für größere \(n\) der analytsichen Lösung immer ähnlicher
\end{itemize}
\end{frame}




\begin{frame}{Maximaler Fehler}



%TO DO ADD GRAPH



\textbf{Beobachtung:}
\begin{itemize}
    \item Anfangs noch kein Muster zu erkennen
    \item Ab \(N \approx 10\) verläuft der Fehler proportional zu \(N^{-2}\)
\end{itemize}
\end{frame}




\begin{frame}{Fehler an den Diskretisierungspunkten}
\end{frame}















\section{Zusammenfassung}

\begin{frame}{Zusammenfassung}
    \textbf{Erkenntnisse aus den Experimenten:}
    \begin{itemize}
        \item Theoretischer Speicherplatzbedarf für \(A\) liegt im vollbesetzten Format in
        \(\mathcal{O}(n^4)\) und im sparse Format in \(\mathcal{O}(n^2)\) 
        \item relative Sparsity liegt 
    \end{itemize}

    \textbf{Vergleich: Analytische vs numerische Lösung}
    \begin{itemize}
        \item Numerische Lösung sieht mit wachsendem \(n\) immer mehr wie die
        die analytische Lösung aus.
    \end{itemize}

    \textbf{Fehleranalyse:}
    \begin{itemize}
        \item Maximaler Fehler zwischen analytischer und numerischer Lösung liegt
        in \(\mathcal{O}(n^{-2})\)
        \item 
    \end{itemize}
\end{frame}



\begin{frame}{Literaturverzeichnis}
    \printbibliography
\end{frame}

\end{document}