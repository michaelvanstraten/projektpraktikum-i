\documentclass{scrartcl}

\usepackage{csquotes}
% Babel provides hyphenation patterns and translations of keywords like 'table
% of contents'
\usepackage[ngerman]{babel}
\usepackage{biblatex}
% Automatic generation of hyperlinks for references and URIs
\usepackage{hyperref}
\usepackage{graphicx}
\usepackage{float}
\usepackage{luatodonotes}

\usepackage{amsmath, amssymb, amsthm, mathtools}

\newcommand\N{\mathbb{N}}
\newcommand\R{\mathbb{R}}
\newcommand\BigO{\mathcal{O}}

\theoremstyle{definition}
\newtheorem{definition}{Definition}
\newtheorem{theorem}{Theorem}

\DeclarePairedDelimiterXPP{\set}[1]{}{\lbrace}{\rbrace}{}{#1}
\DeclarePairedDelimiterXPP{\paren}[1]{}{\lparen}{\rparen}{}{#1}

\newcommand{\definitionU}{
    \[
        u(x) \coloneqq \prod_{l=1}^2 x_l \sin(\kappa \pi x_l)
    \]
}

\newcommand{\definitionF}{
    \begin{align*}
        f(x) \coloneq 2 \kappa \pi
        ( & - x_1 \cos(\kappa \pi x_2) \sin(\kappa \pi x_2) \\
          & + x_2 (\cos(\kappa \pi x_1)
        + \kappa \pi x_1 \sin(\kappa \pi x_1)) \cdot \sin(\kappa \pi x_2))
    \end{align*}
}


\KOMAoptions{
  % Add vertical space between two paragraphs, without indent
  parskip=true,
}

\subject{Bericht}
\titlehead{%
  \begin{minipage}{.7\textwidth}%
  Humboldt-Universit\"at zu Berlin\\
  Mathematisch-Naturwissenschaftliche Fakult\"at\\
  Institut f\"ur Mathematik
  \end{minipage}
}
\title{Lösen des Poisson-Problems mittels SOR-Verfahren}
\author{%
  Eingereicht von M. van Straten und P. Merz
}
\date{\today}

\graphicspath{{./figures/}}

\addbibresource{report.bib}

\begin{document}

\maketitle
\tableofcontents
\cleardoublepage%

%TODO ADD CITATIONS
\section{Einleitung und Motivation}
Im Laufe des Moduls "Projekt Praktikum I" haben wir uns zuerst mit dem Thema
Finite Differenzen beschäftigt. Mit hilfe dieser haben wir das Poisson-Problem
\begin{align}\label{Eq:Poisson}
    -\Delta u & = f \; \text{in} \; \Omega           \\
    u         & = g \; \text{auf} \; \partial \Omega
\end{align}
wobei
\begin{equation*}
    \Delta u = \frac{\partial^2 u}{\partial x_1^2} + \frac{\partial^2 u}{\partial x_2^2}
\end{equation*}
den Laplace-Operator von \(u\) darstellt, für eine gegebene Funktion \(f\)
auf \(\Omega = {(0, 1)}^2\) und \(g = 0\) diskretisiert und anschließend mit
der LU-Zerlegung der entstehenden Diskretisierungsmatrix \(A\) gelöst. Während die Matrix \(A\) zwar dünn besetzt war,
war die LU-Zerlegung dieser nicht dünn besetzt. Aufgrund dessen wollen wir in diesem Bericht eine weitere Art des
Lösens für das entstehende Gleichungssystem betrachten: Das SOR-Verfahren. Dieses Verfahren ist ein iteratives Verfahren
und hat damit den Vorteil, das der Rechenaufwand pro Iteration verhältnismäßig gering ist \cite{AB3}

\section{Theoretische Grundlagen}

\subsection{Poisson-Problem (diskretisiert)}
Durch Diskretisieren des Laplace-Operators und des Gebietes auf \((n-1)^2\)
Gitterpunkte wird Gleichung \ref{Eq:Poisson} zu einem Gleichungssystem in
\((n-1)^2\) Unbekannten. Die Koeffizientenmatrix, die das Gleichungssystem
beschreibt lautet \(h^{-2}A_p\) mit \(h = \frac{1}{n}\) und
\[
    A_p \coloneq \begin{bmatrix}
        C      & -I     & 0      & \cdots & 0      \\
        -I     & C      & -I     & \cdots & 0      \\
        0      & \ddots & \ddots & \ddots & \vdots \\
        \vdots & \ddots & \ddots & \ddots & -I     \\
        0      & \cdots & 0      & -I     & C
    \end{bmatrix},
\]
wobei \(C \in \R^{(n-1) \times (n-1)}\) eine Tridiagonalmatrix ist:
\[
    C \coloneq \begin{bmatrix}
        4      & -1     & 0      & \cdots & 0      \\
        -1     & 4      & -1     & \cdots & 0      \\
        0      & \ddots & \ddots & \ddots & \vdots \\
        \vdots & \ddots & \ddots & \ddots & -1     \\
        0      & \cdots & 0      & -1     & 4
    \end{bmatrix}.
\]\cite{HandoutLU}

\subsection{Iterative Verfahren}
Sei im folgenden immer ein lineares Gleichungssystem
\[Ax=b\]
mit \(A \in \R^{n \times n}\) regulär und \(b \in \R^n\) gegeben.
\begin{definition}\cite{Iterative}
    Ein Iterationsverfahren ist gegeben durch die Abbildung
    \[\phi:\R^n \times \R^n \rightarrow \R^n\]
    mit Iterationsvorschrift
    \[x^{(k+1)}=\phi(x^{(k)},b)\]
    Das Iterationsverfahren heißt linear, falls \(B,C \in \R^{n \times n}\)
    existieren, sodass
    \[\phi(x,b)=Bx+Cb\]
\end{definition}

\begin{definition}\cite{Iterative}
    Eine Verfahrensfunktion heißt konvergent wenn für alle \(b \in \R^n\) und alle \(x_0 \in \R^n\) ein vom Startwert unabhängiger Grenzwert
    \[\tilde{x} = \lim_{k \to \infty} \phi(x^{(k)},b)\]
    existiert\\ Eine Verfahrensfunktion heißt konsistent zur Matrix \(A\),
    falls die Lösung \(\tilde{x}\) des linearen Gleichungssystems ein Fixpunkt
    der Verfahrensfunktion ist, das heißt
    \[\tilde{x} = \phi(\tilde{x},b)\]
\end{definition}
Die Konvergenz und Konsistenz stellen zwei sinnvolle Kriterien dar, die wir von einem iterativen Verfahren erwarten, da diese beiden sicherstellen, dass das Verfahren gegen die Lösung des linearen Gleichungssystems konvergiert, deshalb folgender Satz

\begin{theorem}\cite{Iterative}
    Sei \(\phi(x,b)=Bx+Cb\) eine lineare Verfahrensfunktion, dann gilt:
    \(\phi\) ist genau dann konsistent zur Matrix \(A\), falls
    \[B=I-CA.\]
    Außerdem ist \(\phi\) genau dann konvergent, wenn für den Spektralradius
    von \(B\)
    \[\rho(B)<1\]\label{Thrm:Spectralradius}
    gilt.
\end{theorem}
\begin{proof}
    Siehe A. Meister Numerik linearer Gleichungssysteme 5. Auflage Satz 4.4 und 4.5
\end{proof}
\subsection{Splitting-Verfahren}
Splitting-Verfahren basieren auf dem Zerlegen der Koeffizientenmatrix \(A\) in
zwei Matrizen \(M, N \in \R^{n \times n}\), mit \(M\) invertierbar, sodass
\[A = M-N\]
gilt. Für das lineare Gleichungssystem gilt dann
\begin{align*}
    \iff Ax     & = b                  \\
    \iff (M-N)x & = b                  \\
    \iff Mx     & = Nx + b             \\
    \iff x      & = M^{-1}Nx + M^{-1}b
\end{align*}\cite{SOR}
Das heißt die Lösung des linearen Gleichungssystems ist per Konstruktion ein Fixpunkt des Iterationsverfahrens
\[x^{(k+1)}= M^{-1}Nx^{(k)} + M^{-1}b\]
Also ist dieses Verfahren per Definition konsistent zur Matrix \(A\). \\ Gilt
  zusätzlich \(\rho(M^{-1}N) < 1\), so ist das Verfahren nach Theorem
  \ref{Thrm:Spectralradius} auch konvergent

\subsection{SOR-Verfahren}
Gegeben sei ein lineares Gleichungssystem
\[Ax = b\]
mit \(A \in \mathbb{R}^{n \times n} \quad x,b \in \mathbb{R}^n\). Man zerlege
  A in eine Diagonalmatrix \(D\), eine strikte linke untere Dreiecksmatrix
  \(L\) und eine strikte obere Dreiecksmatrix \(U\), sodass \(A = D -L-U)\) mit
\[
    D = \begin{pmatrix}
        a_{11} & 0      & \cdots & 0      \\
        0      & a_{22} & \cdots & 0      \\
        \vdots & \vdots & \ddots & \vdots \\
        0      & 0      & \cdots & a_{nn}
    \end{pmatrix}, \quad
    L = \begin{pmatrix}
        0       & 0       & \cdots & 0      \\
        -a_{21} & 0       & \cdots & 0      \\
        \vdots  & \vdots  & \ddots & \vdots \\
        -a_{n1} & -a_{n2} & \cdots & 0
    \end{pmatrix}, \quad
    R = \begin{pmatrix}
        0      & -a_{12} & \cdots & -a_{1n} \\
        0      & 0       & \cdots & -a_{2n} \\
        \vdots & \vdots  & \ddots & \vdots  \\
        0      & 0       & \cdots & 0
    \end{pmatrix}
\]

Sei ferner \(\omega \neq 0\). Wähle \[M = \frac{1}{\omega}D-L\]
wobei \(\operatorname{det}D \neq 0\) gelten soll, das heißt \(D\) ist
  invertierbar, und aufgrund der linken unteren Dreiecksstruktur von \(M\) ist
  diese Matrix ebenfalls invertierbar. Damit ist \[N= M-A=(\frac{1-\omega}{\omega})D-U\]

und daher lautet die Iterationsvorschrift
\[x^{(k+1)}=(D - \omega L)^{-1}[(1- \omega)D + \omega U]x^{(k)} + \omega (D-\omega U)^{-1}b\]

Nutzt man die untere Dreiecksstruktur von \(D - \omega L\) aus, kann man
mittels Vorwärtssubstitution die einzelnen Einträge von \(x^{(k+1)}\) berechnen
und es gilt:
\[
    x_i^{(k+1)} = (1 - \omega) x_i^{(k)} + \frac{\omega}{a_{ii}}
    \left( b_i - \sum_{j < i} a_{ij} x_j^{(k+1)} - \sum_{j > i} a_{ij} x_j^{(k)} \right),
    \quad i = 1, 2, \ldots, n.
\]\cite{SOR}

\subsection{Konvergenz des SOR-Verfahren}
\begin{theorem}
    Sei \(A \in \mathbb{R}^n\) symmetrisch und positiv definit, dann gilt aufgrund der Symmetrie
    für die Zerlegung von \(A\), dass
    \[U = L^T.\]
    Daraus folgt: Das SOR-Verfahren konvergiert für \(0 < \omega < 2\) für alle
    Startwerte \(x^{(0)} \in \mathbb{R}^n\).

\end{theorem}

\begin{proof}
    Siehe Numerische Mathematik für Ingenieure und Physiker
    Band 1: Numerische Methoden der Algebra Satz 6.3-2. von W. Törnig
\end{proof}

\subsubsection{Anwendung auf die Matrix \(A_{p}\)}
Die Matrix \(A_p\), die sich durch Diskretisierung des Laplace-Operators ergibt, ist aufgrund ihrer Struktur trivialerweise symmetrisch. Ferner ist sie auch positiv definit\cite{PosDef}.
Also konvergiert das SOR-Verfahren für \(\omega \in (0, 2)\) für diese Matrix.

\section{Experimentelle Untersuchungen}

Für die folgenden Experimente definieren wir die Funktion \definitionU sowie
\definitionF mit \(\kappa = 3\), für welche gelten das
\[
    - \Delta u(x) = f(x)
\]
ist.

\subsection{Lösung des Poisson-Problems}

Ähnlich wie zur Lösung der Poisson-Gleichung via LU-Zerlegung wollen wir
untersuchen, ob und wie sich der maximalen absoluten Fehler entwickelt.
Hierfür wollen wir diesen Grafisch darstellen, was wir mit dem folgenden code
Snippet erreicht haben.

% [TODO] Add code snippet

\begin{figure}[H]
    \centering
    % \includegraphics[width=0.8\textwidth]{error_plot_eps1}
    \caption{Maximaler absoluter Fehler in Abhängigkeit der Iterationen für
        $\epsilon = 10^{-6}$}
    \label{fig:error_plot_eps1}
\end{figure}

Wir sehen das der maximale Fehler für größere \(n\) schneller gegen 0 zu
konvergieren scheint dann allerdings für \(n = \todo{Wert ergänzen}\) wieder
ansteigt. Dies könnte an der Wahl des Abbruchkriteriums liegen, welches wir auf
\(\epsilon = 10^{\todo{Wert ergänzen}}\) gesetzt haben. Für größere \(n\)
könnte es sein das dieses Kriterium zu streng ist und das Verfahren deshalb
abbricht bevor es konvergiert.

Um dies zu überprüfen haben wir dieselbe Untersuchung für \(\epsilon =
10^{\todo{Wert ergänzen}}\)erneut durchgeführt.

\begin{figure}[H]
    \centering
    % \includegraphics[width=0.8\textwidth]{error_plot_eps2}
    \caption{Maximaler absoluter Fehler in Abhängigkeit der Iterationen für
        $\epsilon = 10^{-8}$}
    \label{fig:error_plot_eps2}
\end{figure}

\subsection{Optimale Wahl von \(\epsilon\)}\label{sec:optimal_epsilon}

Wie wir in den vorherigen Experimenten gesehen haben, hat die Wahl von
\(\epsilon\) einen großen Einfluss auf die Konvergenz des SOR-Verfahrens.
Deshalb wollen wir nun untersuchen wie sich der maximale absolute Fehler in
Abhängigkeit von \(\epsilon\) entwickelt.

Hierfür haben wir uns den maximalen absoluten Fehler für
\[
    \epsilon = h^k \quad \text{mit} \quad
    k \in \set{-2, 0, 2, 4, 6}
\]
angesehen.

Der code Snippet um dies zu erreichen sieht wie folgt aus:

% [TODO] Add code snippet

\begin{figure}[H]
    \centering
    % \includegraphics[width=0.8\textwidth]{optimal_epsilon}
    \caption{Maximaler absoluter Fehler für verschiedene \(\epsilon\)}
    \label{fig:optimal_epsilon}
\end{figure}

Wir sehen das mit steigendem \(k\) der maximale Fehler abnimmt. Für \(k = 4\)
und \(k = 6\) scheint der Graph allerdings nahezu identisch zu sein.

Dies ist auch zu erwarten, da für kleinere \(\epsilon\) das Verfahren zu früh
abbricht und für größere \(\epsilon\) das Verfahren zu lange läuft. Der Fakt,
dass der Graph für \(\epsilon = h^{-6}\) gleich zu dem für \(\epsilon =
h^{-4}\) ist, lässt folgern das \(\epsilon = h^{-4}\) die optimale Wahl ist da
weitere Iterationen keinen zusätzlichen Nutzen bringen.

\subsection{Optimaler Relaxationsparameter \(\omega\)}

% [TODO]

\subsection{Vergleich mit dem LU-Verfahren}

Da wir bereits das LU-Verfahren zu Lösung des Poisson-Problems untersucht haben
scheint es sinnvoll die beiden Verfahren miteinander zu vergleichen.

\subsubsection{Konvergenz verhalten}

Hierfür betrachten wir zunächst inwiefern beide Verfahren in der Lage sind, die
exakte Lösung zu approximieren. Hierfür haben wir wiederum den maximalen
absoluten Fehler in Abhängigkeit der Iterationen betrachtet. Für das
SOR-Verfahren haben wir \(\epsilon = 10^{-8}\) fest sowie optimal wie in
\autoref{sec:optimal_epsilon} bestimmt.

Um den folgenden Graphen zu erstellen haben wir den folgenden code Snippet
verwendet:

% [TODO] Add code snippet

\begin{figure}[H]
    \centering
    % \includegraphics[width=0.8\textwidth]{convergence-comparison}
    \caption{Vergleich des maximalen absoluten Fehlers für das LU- und
        SOR-Verfahren}
    \label{fig:convergence-comparison}
\end{figure}

Wir beobachten das beide Verfahren bis \(n = \todo{Wert ergänzen}\) sehr
ähnlich konvergieren. Ab diesem Punkt steigt der maximale Fehler für das
SOR-Verfahren mit \(\epsilon = 10^{-8}\) wieder an. Ab \(n = \todo{Wert
ergänzen}\) geschieht für das SOR-Verfahren mit optimalen \(\epsilon\)
dasselbe.

\subsubsection{Laufzeitverhalten}

Ein wesentlicher Faktor bei der Wahl des Verfahrens ist dessen Laufzeit. Um
dies zu untersuchen haben wir die Laufzeit beider Verfahren in Abhängigkeit der
Diskretisierung Feinheit \(n\) betrachtet. Das SOR-Verfahren haben wir dabei
mit optimalen \(\epsilon\) optimalem \(\omega\) betrachtet. Die dargestellten
Laufzeiten sind dabei gemittelt über \(\todo{Wert ergänzen}\) Durchläufe, um
Beeinflussungen durch andere Prozesse zu minimieren.

Der code Snippet um dies zu erreichen sieht wie folgt aus:

% [TODO] Add code snippet

\begin{figure}[H]
    \centering
    % \includegraphics[width=0.8\textwidth]{runtime-comparison}
    \caption{Vergleich der Laufzeiten des LU- und SOR-Verfahrens}
    \label{fig:runtime-comparison}
\end{figure}

\section{Auswertung}

% [TODO]

\section{Zusammenfassung}

% [TODO]

\printbibliography%

\end{document}
