\documentclass{scrartcl}

\usepackage{csquotes}
% Babel provides hyphenation patterns and translations of keywords like 'table
% of contents'
\usepackage[ngerman]{babel}
% Provides commands for type-setting mathematical formulas
\usepackage{amsmath}
% Provides additional symbols
\usepackage{amssymb}
% Provides environments for typical math text structure
\usepackage{amsthm}
\usepackage{biblatex}
% Automatic generation of hyperlinks for references and URIs
\usepackage{hyperref}
\usepackage{mathtools}
\usepackage{graphicx}
\usepackage{float}


\KOMAoptions{
  % Add vertical space between two paragraphs, without indent
  parskip=true,
}

\subject{Bericht}
\titlehead{%
  \begin{minipage}{.7\textwidth}%
  Humboldt-Universit\"at zu Berlin\\
  Mathematisch-Naturwissenschaftliche Fakult\"at\\
  Institut f\"ur Mathematik
  \end{minipage}
}
\title{Lösen des Poisson-Problems mittels SOR-Verfahren}
\author{%
  Eingereicht von M. van Straten und P. Merz
}
\date{\today}

\newcommand\N{\mathbb{N}}
\newcommand\R{\mathbb{R}}
\newcommand\BigO{\mathcal{O}}

\theoremstyle{definition}
\newtheorem{definition}{Definition}

\newtheorem{theorem}{Theorem}

\DeclarePairedDelimiterXPP{\set}[1]{}{\lbrace}{\rbrace}{}{#1}
\DeclarePairedDelimiterXPP{\paren}[1]{}{\lparen}{\rparen}{}{#1}

\graphicspath{{./figures/}}

\addbibresource{report.bib}

\begin{document}

\maketitle
\tableofcontents
\cleardoublepage%


%TODO ADD CITATIONS
\section{Einleitung und Motivation}
Im Laufe des Moduls "Projekt Praktikum I" haben wir uns zuerst mit dem Thema Finite Differenzen
beschäftigt. Mit hilfe dieser haben wir das Poisson-Problem
\begin{align}\label{Eq:Poisson}
    -\Delta u & = f \; \text{in} \; \Omega           \\
    u         & = g \; \text{auf} \; \partial \Omega
\end{align}
wobei
\begin{equation*}
    \Delta u = \frac{\partial^2 u}{\partial x_1^2} + \frac{\partial^2 u}{\partial x_2^2}
\end{equation*}
den Laplace-Operator von \(u\) darstellt, für eine gegebene Funktion \(f\)
auf \(\Omega = {(0, 1)}^2\) und \(g = 0\) diskretisiert und anschließend mit 
der LU-Zerlegung der entstehenden Diskretisierungsmatrix \(A\) gelöst. Während die Matrix \(A\) zwar dünn besetzt war,
war die LU-Zerlegung dieser nicht dünn besetzt. Aufgrund dessen wollen wir in diesem Bericht eine weitere Art des
Lösens für das entstehende Gleichungssystem betrachten: Das SOR-Verfahren. Dieses Verfahren ist ein iteratives Verfahren
und hat damit den Vorteil, das der Rechenaufwand pro Iteration verhältnismäßig gering ist \cite{AB3}




\section{Theoretische Grundlagen}


\subsection{Poisson-Problem (diskretisiert)}
Durch Diskretisieren des Laplace-Operators und des Gebietes auf \((n-1)^2\) Gitterpunkte wird Gleichung \ref{Eq:Poisson} zu einem Gleichungssystem in \((n-1)^2\) Unbekannten. Die Koeffizientenmatrix, die das Gleichungssystem beschreibt lautet \(h^{-2}A_p\) mit \(h = \frac{1}{n}\) und
\[
    A_p \coloneq \begin{bmatrix}
        C      & -I     & 0      & \cdots & 0      \\
        -I     & C      & -I     & \cdots & 0      \\
        0      & \ddots & \ddots & \ddots & \vdots \\
        \vdots & \ddots & \ddots & \ddots & -I     \\
        0      & \cdots & 0      & -I     & C
    \end{bmatrix},
\]
wobei \(C \in \R^{(n-1) \times (n-1)}\) eine Tridiagonalmatrix ist:
\[
    C \coloneq \begin{bmatrix}
        4      & -1     & 0      & \cdots & 0      \\
        -1     & 4      & -1     & \cdots & 0      \\
        0      & \ddots & \ddots & \ddots & \vdots \\
        \vdots & \ddots & \ddots & \ddots & -1     \\
        0      & \cdots & 0      & -1     & 4
    \end{bmatrix}.
\]\cite{HandoutLU}

\subsection{Iterative Verfahren}
Sei im folgenden immer ein lineares Gleichungssystem
\[Ax=b\]
mit \(A \in \R^{n \times n}\) regulär und \(b \in \R^n\) gegeben.
\begin{definition}\cite{Iterative}
Ein Iterationsverfahren ist gegeben durch die Abbildung
\[\phi:\R^n \times \R^n \rightarrow \R^n\]
mit Iterationsvorschrift
\[x^{(k+1)}=\phi(x^{(k)},b)\]
Das Iterationsverfahren heißt linear, falls \(B,C \in \R^{n \times n}\) existieren, sodass
\[\phi(x,b)=Bx+Cb\]
\end{definition}


\begin{definition}\cite{Iterative}
Eine Verfahrensfunktion heißt konvergent wenn für alle \(b \in \R^n\) und alle \(x_0 \in \R^n\) ein vom Startwert unabhängiger Grenzwert
\[\tilde{x} = \lim_{k \to \infty} \phi(x^{(k)},b)\]
existiert\\
Eine Verfahrensfunktion heißt konsistent zur Matrix \(A\), falls die Lösung \(\tilde{x}\) des linearen Gleichungssystems ein Fixpunkt der Verfahrensfunktion ist, das heißt
\[\tilde{x} = \phi(\tilde{x},b)\]
\end{definition}
Die Konvergenz und Konsistenz stellen zwei sinnvolle Kriterien dar, die wir von einem iterativen Verfahren erwarten, da diese beiden sicherstellen, dass das Verfahren gegen die Lösung des linearen Gleichungssystems konvergiert, deshalb folgender Satz

\begin{theorem}\cite{Iterative}
    Sei \(\phi(x,b)=Bx+Cb\) eine lineare Verfahrensfunktion, dann gilt:
    \(\phi\) ist genau dann konsistent zur Matrix \(A\), falls
    \[B=I-CA.\]
    Außerdem ist \(\phi\) genau dann konvergent, wenn für den Spektralradius von \(B\)
    \[\rho(B)<1\]\label{Thrm:Spectralradius}
    gilt.   
\end{theorem}
\begin{proof}
    Siehe A. Meister Numerik linearer Gleichungssysteme 5. Auflage Satz 4.4 und 4.5
\end{proof}
\subsection{Splitting-Verfahren}
Splitting-Verfahren basieren auf dem Zerlegen der Koeffizientenmatrix \(A\) in zwei Matrizen
\(M, N \in \R^{n \times n}\), mit \(M\) invertierbar, sodass 
\[A = M-N\]
gilt.
Für das lineare Gleichungssystem gilt dann
\begin{align*}
        \iff Ax &= b \\
    \iff (M-N)x &= b \\
        \iff Mx &= Nx + b\\
         \iff x &= M^{-1}Nx + M^{-1}b
\end{align*}\cite{SOR}
Das heißt die Lösung des linearen Gleichungssystems ist per Konstruktion ein Fixpunkt des Iterationsverfahrens 
\[x^{(k+1)}= M^{-1}Nx^{(k)} + M^{-1}b\]
Also ist dieses Verfahren per Definition konsistent zur Matrix \(A\). \\
Gilt zusätzlich \(\rho(M^{-1}N) < 1\), so ist das Verfahren nach Theorem \ref{Thrm:Spectralradius}
auch konvergent


\subsection{SOR-Verfahren}
Gegeben sei ein lineares Gleichungssystem 
\[Ax = b\]
mit \(A \in \mathbb{R}^{n \times n} \quad x,b \in \mathbb{R}^n\).
Man zerlege A in eine Diagonalmatrix \(D\), eine strikte linke untere Dreiecksmatrix 
\(L\) und eine strikte obere Dreiecksmatrix \(U\), sodass
\(A = D -L-U)\) mit
\[
D = \begin{pmatrix}
a_{11} & 0 & \cdots & 0 \\
0 & a_{22} & \cdots & 0 \\
\vdots & \vdots & \ddots & \vdots \\
0 & 0 & \cdots & a_{nn}
\end{pmatrix}, \quad
L = \begin{pmatrix}
0 & 0 & \cdots & 0 \\
-a_{21} & 0 & \cdots & 0 \\
\vdots & \vdots & \ddots & \vdots \\
-a_{n1} & -a_{n2} & \cdots & 0
\end{pmatrix}, \quad
R = \begin{pmatrix}
0 & -a_{12} & \cdots & -a_{1n} \\
0 & 0 & \cdots & -a_{2n} \\
\vdots & \vdots & \ddots & \vdots \\
0 & 0 & \cdots & 0
\end{pmatrix}
\]


Sei ferner \(\omega \neq 0\).
Wähle \[M = \frac{1}{\omega}D-L\] 
wobei \(\operatorname{det}D \neq 0\) gelten soll, das heißt \(D\) ist invertierbar, 
und aufgrund der linken unteren Dreiecksstruktur von \(M\)
ist diese Matrix ebenfalls invertierbar.
Damit ist \[N= M-A=(\frac{1-\omega}{\omega})D-U\]

und daher lautet die Iterationsvorschrift
\[x^{(k+1)}=(D - \omega L)^{-1}[(1- \omega)D + \omega U]x^{(k)} + \omega (D-\omega U)^{-1}b\]








Nutzt man die untere Dreiecksstruktur von \(D - \omega L\) aus, kann man mittels Vorwärtssubstitution die einzelnen 
Einträge von \(x^{(k+1)}\) berechnen und es gilt:
\[
x_i^{(k+1)} = (1 - \omega) x_i^{(k)} + \frac{\omega}{a_{ii}} 
\left( b_i - \sum_{j < i} a_{ij} x_j^{(k+1)} - \sum_{j > i} a_{ij} x_j^{(k)} \right), 
\quad i = 1, 2, \ldots, n.
\]\cite{SOR}

\subsection{Konvergenz des SOR-Verfahren}
\begin{theorem}
Sei \(A \in \mathbb{R}^n\) symmetrisch und positiv definit, dann gilt aufgrund der Symmetrie
für die Zerlegung von \(A\), dass
\[U = L^T.\]
Daraus folgt: Das SOR-Verfahren konvergiert für \(0 < \omega < 2\) für alle Startwerte \(x^{(0)} \in \mathbb{R}^n\).

\end{theorem}

\begin{proof}
   Siehe Numerische Mathematik für Ingenieure und Physiker
Band 1: Numerische Methoden der Algebra Satz 6.3-2. von W. Törnig 
\end{proof}


\subsubsection{Anwendung auf die Matrix \(A_{p}\)}
Die Matrix \(A_p\), die sich durch Diskretisierung des Laplace-Operators ergibt, ist aufgrund ihrer Struktur trivialerweise symmetrisch. Ferner ist sie auch positiv definit\cite{PosDef}.
Also konvergiert das SOR-Verfahren für \(\omega \in (0, 2)\) für diese Matrix.


\section{Experimentelle Untersuchungen}
Für unsere experimentellen Untersuchungen wurde, wie für die Untersuchungen zur LU-Zerlegung, die Funktion
\begin{equation*}
u(x) := x_1 \cdot \sin(3 \cdot x_1) \cdot x_2 \cdot \sin(3 \cdot x_2)
\end{equation*}
betrachtet. Für diese Funktion gilt:
\begin{equation*}
f(x) := -6 \begin{pmatrix}
x_1 \cdot \cos(3 \cdot x_2) \cdot \sin(3 \cdot x_1) \\ 
x_2 \cdot \cos(3 \cdot x_1) \cdot \sin(3 \cdot x_2) \\ 
-3 \cdot x_1 \cdot x_2 \sin(3 \cdot x_1) \cdot \sin(3 \cdot x_2)
\end{pmatrix}
= -\Delta u(x).
\end{equation*}

%Zuerst haben wir mithilfe des SOR-Verfahren für Werte von n zwische PLACEHOLDER UND %PLACEHOLDER für die Funktion \(f\) gelöst und dann mit der Funktion \(u\), die für \(f\) die %analytische Lösung des Problems darstellt, den maximalen Fehler zwischen numerischer und %analytischer Lösung im log-log-Plot dargestellt.

%TODO ADD DIAGRAM

%b)
%TODO ADD DESCRIPTION AND DIAGRAM


%TODO MAKE THIS WAY BETTER
%Ferner haben wir den optimalen Relaxationsfaktor, mithilfe von PLACEHOLDER ermittelt





\subsection{Beobachtungen}



\section{Auswertung}



\section{Zusammenfassung}


\printbibliography%

\end{document}
