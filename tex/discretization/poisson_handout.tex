\documentclass{scrartcl}
\usepackage{scrhack}  % resolves deprecated warnings in context of float and KOMA

%%% GENERAL PACKAGES %%%%%%%%%%%%%%%%%%%%%%%%%%%%%%%%%%%%%%%%%%%%%%%%%%%%%%%%%%
% inputenc allows the usage of non-ascii characters in the LaTeX source code
\usepackage[utf8]{inputenc}


%%% KOMA OPTIONS %%%%%%%%%%%%%%%%%%%%%%%%%%%%%%%%%%%%%%%%%%%%%%%%%%%%%%%%%%%%%%
\KOMAoptions{
  % font size in pt
  fontsize=12,
  % parameter for size of type area, cf. KOMA script documentation
  DIV=13,
  % option for two-sided documents : false, semi, true
  twoside=false,
  % option for paragraph skip :
  %   false - indent, without vertical space between two paragraphs
  %   true  - vertical space between two paragraphs, without indent
  parskip=true,
  % option to generate a draft version
  draft=false
}


%%% TITLEPAGE %%%%%%%%%%%%%%%%%%%%%%%%%%%%%%%%%%%%%%%%%%%%%%%%%%%%%%%%%%%%%%%%%
% general description of the type of the document
\subject{Bericht Handout}
% head line of the title page, here with department and logo
\titlehead{%
  \begin{minipage}{.7\textwidth}%
  Humboldt-Universit\"at zu Berlin\\
  Mathematisch-Naturwissenschaftliche Fakult\"at\\
  Institut f\"ur Mathematik
  \end{minipage}
}
% title of the document
\title{Poisson-Problem}
% optional subtitle
%\subtitle{Draft from~\today}
% information about the author
\author{%
  Eingereicht von M. van Straten und P. Merz
}
% date of submission
\date{\today}



%%% LANGUAGE %%%%%%%%%%%%%%%%%%%%%%%%%%%%%%%%%%%%%%%%%%%%%%%%%%%%%%%%%%%%%%%%%%
% babel provides hyphenation patterns and translations of keywords like 'table
% of contents'
\usepackage[ngerman]{babel}

\usepackage{mathtools}
%%% MATH %%%%%%%%%%%%%%%%%%%%%%%%%%%%%%%%%%%%%%%%%%%%%%%%%%%%%%%%%%%%%%%%%%%%%%
% amsmath provides commands for type-setting mathematical formulas
\usepackage{amsmath}
% amssymb provides additional symbols
\usepackage{amssymb}
% HINT
% Use http://detexify.kirelabs.org/classify.html to find unknown symbols!

\usepackage{cite}

%%% MATH ENVIRONMENTS %%%%%%%%%%%%%%%%%%%%%%%%%%%%%%%%%%%%%%%%%%%%%%%%%%%%%%%%%
% amsthm provides environments for typical math text structure
\usepackage{amsthm}
\theoremstyle{remark}
\newtheorem*{bemerkung}{Bemerkung}


%%% COLORS %%%%%%%%%%%%%%%%%%%%%%%%%%%%%%%%%%%%%%%%%%%%%%%%%%%%%%%%%%%%%%%%%%%%
% define own colors and use colored text
\usepackage[pdftex,svgnames,hyperref]{xcolor}


%%% FONT %%%%%%%%%%%%%%%%%%%%%%%%%%%%%%%%%%%%%%%%%%%%%%%%%%%%%%%%%%%%%%%%%%%%%%
% choose latin modern font (a good general purpose font)
\usepackage{lmodern}
% fontenc and microtype improve the appearance of the font
\usepackage[T1]{fontenc}
\usepackage{microtype}


%%% HYPERLINKS %%%%%%%%%%%%%%%%%%%%%%%%%%%%%%%%%%%%%%%%%%%%%%%%%%%%%%%%%%%%%%%%
% automatic generation of hyperlinks for references and URIs
\usepackage{hyperref}


%%% GRAPHICAL ELEMENTS %%%%%%%%%%%%%%%%%%%%%%%%%%%%%%%%%%%%%%%%%%%%%%%%%%%%%%%%
% provides commands to include graphics
\usepackage{graphicx}


%%% TABLES %%%%%%%%%%%%%%%%%%%%%%%%%%%%%%%%%%%%%%%%%%%%%%%%%%%%%%%%%%%%%%%%%%%%
% provides commands for good-looking tables
\usepackage{booktabs}

%%%% more fonts: e.g. fraktur
\usepackage{yfonts}

%%% Code Listings %%%%%%%%%%%%%%%%
% provides commands for including code (python, latex, ...)
\usepackage{listings}
\definecolor{keywords}{RGB}{255,0,90}
\definecolor{comments}{RGB}{0,0,113}
\definecolor{red}{RGB}{160,0,0}
\definecolor{green}{RGB}{0,150,0}
\lstset{language=Python, 
        basicstyle=\ttfamily\small, 
        keywordstyle=\color{keywords},
        commentstyle=\color{comments},
        stringstyle=\color{red},
        showstringspaces=false,
        identifierstyle=\color{green},
        }

%%%% Grafiken zeichnen mit tikz
\usepackage{tikz}
% Definitionen f\"ur das Flowchart
\usetikzlibrary{shapes.geometric, arrows, shapes.multipart}
\tikzstyle{startstop} = [ellipse, minimum width=2.5cm, minimum height=0.8cm, 
	align = left,text centered, draw=black, fill=red!30]
\tikzstyle{io} = [trapezium, trapezium left angle=70, trapezium right angle=110,
	minimum width=2.5cm, minimum height=0.8cm, align = left, text centered, 
	draw=black, fill=blue!30]
\tikzstyle{process} = [rectangle, minimum width=2.5cm, minimum height=0.8cm, 
	align = left, text centered, draw=black, fill=gray!30]
\tikzstyle{err} = [rectangle, minimum width=2.5cm, minimum height=0.8cm, 
	align = left, text centered, draw=black, fill=red!10,text width=3cm]
\tikzstyle{subrout} = [rectangle split,rectangle split horizontal,
	minimum width=2.5cm, minimum height=0.8cm,align = left, text centered, 
	draw=black, fill=orange!30,rectangle split parts=3]
\tikzstyle{decision} = [diamond,shape aspect=2.5, minimum width=2.5cm, 
	minimum height=0.8cm, align = left, text centered, draw=black, 
	fill=green!30]
\tikzstyle{arrow} = [thick,->,>=stealth]




%%% MY USER-DEFINED COMMANDS %%%%%%%%%%%%%%%%%%%%%%%%%%%%%%%%%%%%%%%%%%%%%%%%%%
% define your own commands and abbreviations with \newcommand
% or \DeclareMathOperator
% e.g. try \R and \Span in math mode
\newcommand\R{\mathbb R}
\DeclareMathOperator\Span{span}


%%% BEGIN OF DOCUMENT %%%%%%%%%%%%%%%%%%%%%%%%%%%%%%%%%%%%%%%%%%%%%%%%%%%%%%%%%
% the type-setting of the document starts here
\begin{document}

% generating the title page
\maketitle
% generating the table of contents (requires to run pdflatex twice!)
\tableofcontents
% start a new page
\cleardoublepage


%%% BEGIN OF CONTENT %%%%%%%%%%%%%%%%%%%%%%%%%%%%%%%%%%%%%%%%%%%%%%%%%%%%%%%%%%
\section{Einleitung und Motivation}
Die Poisson-Gleichung, benannt nach Siméon Denis Poisson, ist eine elliptische partielle Differentialgleichung, 
die in vielen Bereichen des Ingenieurswesens und der Physik, wie zum Beispiel in der Gravitationstheorie, der Elektrostatik oder 
der Fluiddynamik, Anwendung findet.\cite{Poisson} Aufgrund der Tatsache, dass es oftmals sehr schwer ist solche Differentialgleichungen analytisch zu lösen
und dass wir in der letzten Aufgabenreihe die Approximation von Ableitungen erster und zweiter Ordnung betrachtet haben, möchten wir für
feste vorgegebene Randbedingungen numerisch eine Approximation der Lösung der Poisson-Gleichung bestimmen und mit der analytischen Lösung
vergleichen.


\section{Theorie}
\subsection{Poisson-Problem} 
Das Poisson-Problem beschreibt die Suche nach einer Funktion \(u \in C^{2}(\mathbb{R}^{2};\mathbb{R})\), sodass für ein gegebenes Gebiet
\(\Omega\ \subset \mathbb{R}^{2}\) mit Rand \(\partial \Omega\) und zwei gebenen Funktionen \(f \in C(\Omega;\mathbb{R})\) und
\(g \in C(\partial \Omega;\mathbb{R})\) folgendes gilt:
\begin{align*}
-\Delta u &= f \text{       in } \Omega  \\
        u &= g \text{       auf } \partial \Omega
\end{align*}
wobei \(\Delta u = \frac{\partial^{2}u}{\partial x_{1}^{2}} + \frac{\partial^{2}u}{\partial x_{2}^{2}}\) den Laplace-Operator für
\(u\) bezeichnet. \cite{PPI_Poisson} \\
In diesem Handout werden wir den Fall \(\Omega = (0, 1)^{2}\) und \(g = 0\) betrachten.



\subsection{Diskretisierung von \(\Omega\)}
Um \(\Omega = (0,1)^{2}\) zu diskretisieren teilen wir dieses in ein Gitter von Punkten auf.
Hierfür zerlege man das Intervall \((0,1)\) in \(n\) gleich lange Teilintervalle der Länge \(\frac{1}{n}\)
und erhält die Menge \(X_{1} = \{\frac{j}{n}: 1 \leqslant j \leqslant n - 1\} \).
Die Gitterpunkte, also Diskretisierungspunkte ergeben sich dann wie folgt:
\[X = X_{1} \times X_{1} = \{(\frac{j}{n}, \frac{k}{n}): 1 \leqslant j,k \leqslant n - 1\} \] \cite{PPI_Poisson}


\subsection{Diskretisierung des Laplace-Operators}
Mithilfe von finiten Differenzen, die bereits im ersten Hausaufgabenteil bearbeitet wurden, lassen sich die zweiten partiellen Ableitungen 
nach einer Variable dann wie folgt approximieren:
\begin{align*}
  \frac{\partial^{2}u}{\partial x_{1}^{2}}(v, w) &\approx \frac{u(v + h, w) - 2u(v, w) + u(v - h, w)}{h^{2}} \\
  \frac{\partial^{2}u}{\partial x_{1}^{2}}(v, w) &\approx \frac{u(v, w + h) - 2u(v, w) + u(v, w - h)}{h^{2}}
\end{align*}
Für \(h = \frac{1}{n}\) mit \(n \in \mathbb{N}^{+}\) und \((v, w) \in X\). \\
Also lässt sich der Laplace-Operator wie folgt approximieren
\begin{align*}
  \Delta u &= \frac{\partial^{2}u}{\partial x_{1}^{2}} + \frac{\partial^{2}u}{\partial x_{2}^{2}} \\
           &\approx \frac{u(v + h, w) + u(v, w + h) - 4u(v, w) + u(v - h, w) + u(v, w - h)}{h^2}
\end{align*}
Dieser diskrete Laplace-Operator von \(u\) wird mit \(\Delta_{h}u\) bezeichnet
\subsection{Aufstellen des linearen Gleichungssystems}
Gesucht ist nun eine Lösung \(\hat{u}\) der diskretisierten partiellen Differentialgleichung
\begin{align*}
  -\Delta_{h} u(x) &= f \text{       für } x \in X   \\
              u(x) &= 0 \text{       auf } \partial \Omega
\end{align*}
an den \(N\) Gitterpunkten. Somit erhält man \(N\) Gleichungen, die zu lösen sind. Um diese Gleichungen zu ordnen
haben wir für \(x = (x_{1}, x_{2}), y = (y_{1}, y_{2} \in X)\) die Ordnung
\[x <_{X} y \iff x_{1}n + x_{2}n^{2} < y_{1}n + y_{2}n^{2}\]
verwendet. Diese Ordnung induziert eine Bijektion
\begin{align*}
  idx: \{1, \cdots ,n - 1\}^{2} &\longrightarrow \{1, \cdots ,N \} \\ 
                          (j, k)&\longmapsto (k - 1)(n - 1) + j
\end{align*}
die wir benutzt haben um einem Diskretisierungspunkt seine Gleichungsnummer zuzuweisen. Die Matrix, die das Gleichungssystem beschreibt ist dann
\(h^{-2}A\), wobei
\[
\begin{array}{ccc}
  \scalebox{0.9}{$
  \begin{array}{l}
    A \coloneq \begin{bmatrix}
    C & -I & 0 & \cdots & 0 \\
    -I & C & -I & \cdots & 0 \\
    0 & \ddots & \ddots & \ddots & \vdots \\
    \vdots & \ddots & \ddots & \ddots & -I \\
    0 & \cdots & 0 & -I & C
    \end{bmatrix}
    \in \mathbb{R}^{N \times N}
  \end{array}
  $}
  &
  \text{und}
  &
  \scalebox{0.9}{$
  \begin{array}{l}
    C \coloneq \begin{bmatrix}
    4 & -1 & 0 & \cdots & 0 \\
    -1 & 4 & -1 & \cdots & 0 \\
    0 & \ddots & \ddots & \ddots & \vdots \\
    \vdots & \ddots & \ddots & \ddots & -1 \\
    0 & \cdots & 0 & -1 & 4
    \end{bmatrix}
    \in \mathbb{R}^{(n-1) \times (n-1)}
  \end{array}
  $}
\end{array}
\] \cite{PPI_Poisson} 

\subsection{LU-Zerlegung einer Matrix}
Sei \(A \in \mathbb{R}^{n \times n}\), dann existieren \(P, L, U \in \mathbb{R}^{n \times n}\), 
wobei \(P\) eine Permutationsmatrix, \(L\) eine linke untere Dreiecksmatrix und \(U\) eine rechte obere
Dreiecksmatrix ist mit \(A = PLU\).  \cite{LU}
Da \(P\) Permutationsmatrix ist, gilt \(P^{-1} = P^{T}\).
Liegt nun ein lineares Gleichungssystem \(Ax = b\) vor, so gilt:
\begin{align*}
  Ax &= b \\
  \iff PLUx &= b \\
  \iff LUx  &= P^{T}b
\end{align*}
Mit der Substitution \(Ux = z\) lässt sich \(Lz = P^{T}b\) nach \(z\) mittels Vorwärtssubstitution in \(\mathcal{O}(n^{2})\) lösen.
Danach kann man \(Ux = z\) mittels Rückwärtssubstitution nach \(x\) in \(\mathcal{O}(n^{2})\) lösen und erhält damit die Lösung des
ursprünglichen Gleichungssystems.

\subsection{Vergleich: A als vollbesetzte und sparse Matrix}
Die Matrix \(A\) besitzt viele Nulleinträge; für größere Werte von \(n\) wird das Speichern
der Matrix als vollbesetzte Matrix sehr ineffizient. Daher haben wir Sparse-Matrizen verwendet.
Sparse-Matrizen speichern nur die von Null verschiedenen Einträge mitsamt ihren Koordinaten innerhalb 
der Matrix. Für einen nicht-Nulleintrag werden also drei Einträge gespeichert.  



\section{Experimente}
Für unsere Experimente haben wir uns die Funktion \(u(x) = x_{1}\sin(3x_{1})x_{2}\sin(3x_{2})\) vorgegeben. Für diese Funktion gilt:
\[-\Delta u(x) = -6(x_{1}\cos(3x_{2})\sin(3x_{1}) + x_{2}\cos(3x_{1})\sin(3x_{2}) - 3x_{1}x_{2}\sin(3x_{1})\sin(3x_{2})) = f(x)\] 
Für diese Funktion \(f\) haben wir eine numerische \(\hat{u}\) des Poisson-Problems, für die am Anfang gegeben Startbedinungen, bestimmt und sie 
mit der analytischen Lösung \(u\) verglichen.

Der erste Teil unserer Experimente richtet sich an die Matrix \(A\) in verschiedenen Speicherformaten.
Zuerst haben wir die Sparsity der Matrix \(A\) und ihrer LU-Zerlegung betrachtet.

%TO DO ADD DIAGRAMS FOR SPARSITY

Zudem haben wir auch den theoretischen Speicherbedarf der Matrix \(A\) im vollbesetzten und sparse Format untersucht.

%TO DO ADD DIAGRAM FOR MEMORY COMPLEXITY

Im nächsten Teil unserer Experimente haben wir die \(\hat{u}\) mit \(u\) verglichen.
Dazu haben wir zunächst beide Funktion für \(n = 4, 11, 128\) geplottet.

%TO DO ADD ANALYTICAL SOLUTION AND NUMERICAL SOLUTION PLOTS

Ferner haben wir den Fehler zwischen der approximierten und tatsächlichen Lösung untersicht, indem wir zuerst den maximalen Fehler
in Abhängigkeit von \(n\) geplottet haben und haben dann die Fehler an den einzelnen Diskretisierungspunkten in 3D, sowie Heatmap-Format
geplottet.

%TO DO ADD ERROR VS NUMBER OF DISCRETIZATION POINTS AND 3D/HEATMAP PLOTS OF ERROR


\subsection{Beobachtungen}

%TO DO ADD REFS TO ABBILDUNGEN

\begin{itemize}
  \item In Abbildung 1 
        ist zu sehen, dass, für größer werdende \(n\), die Anzahl an Gesamteinträgen in \(A\) für sehr viel schneller steigt, als die Anzahl an nicht-Null-Einträgen,
        sowie dass die relative Anzahl an nicht-Null-Einträgen gegen 0 geht.
        Auch für die LU-Zerlegung ist das selbe Verhalten zu erkennen.

        %TO DO MAYBE ADD EXPLANATION OF > 0.001 (by abs)

  \item In Abbildung 2
        ist der theoretischen Speicherplatzbedarf dargestellt. Für größer werdende Werte von \(n\) steigt der Speicherplatzbedarf für das raw-Format
        stärker an, als das CRS-Format, also das Format um \(A\) als sparse-Matrix zu speichern.

  \item In Abbildung 3
        ist der Vergleich der analytischen Lösung mit den approxmierten Lösungen für verschiedene Werte von \(n\) dargestellt.
        Es ist zu beobachten, dass, für größer werdende Werte von \(n\), die approximierte Lösung sich der analytischen Lösung immer mehr annähert.

  \item In Abbildung 4
        ist der maximale Fehler zwischen der numerischen und analytischen Lösung zu sehen.
        Man kann beobachten, dass der Fehler für größer werdende Werte von \(n\) immer kleiner wird.
        Für \(n = 5000\) liegt er bei ca. \(10^{-3}\) und
        für \(n = 10000\) bei ca. \(6 \cdot 10^{-2}\). 


  \item In Abbildung 5
        ist der Fehler an den einzelnen Diskretisierungspunkten, im 3D und Heatmap Plot, zu sehen. \\
        Für \(n = 4\) liegt der Fehler zwischen ungefähr 0.04 und 0.14, \\
        für \(n = 11\) zwischen ungefähr 0.005 und 0.035, und \\
        für \(n = 128\) zwischen ungefähr 0.00005 und 0.00030. \\
        Es ist also eine Verringerung des Fehlers an den einzelnen Diskretisierungspunkten für größer werdende \(n\) zu beobachten.
\end{itemize}



\section{Auswertung}

%TO DO ALSO ADD REFS HERE

\begin{itemize}
  \item Der stärkere Anstieg in Abbildung 1 der Matrix \(A\) im vollbesetzten Format verglichen mit dem sparse Format
        entspricht den Erwartungen, aufgrund der Struktur der Matrix und ihren vielen Nulleinträgen.

        Auch der stärke Anstieg des theoretischen Speicherbedarfs ion Abbildung 2 ist wie im Theorie Teil vorhergesagt, denn obwohl drei
        Einträge für jeden Matrixeintrag im sparse Format gespeichert werden müssen, steigt die Anzahl der Nulleinträge viel 
        schneller und ist daher kostenaufwändiger.

  \item Der graphische Vergleich in Abbildung 3 zeigt, dass die numerische Lösung für größere Werte von \(n\) immer besser
        mit der analytischen Lösung übereinstimmt.

        Dies wird ferner von Abbildung 4 bestätigt, denn der maximale Fehler zwischen der numerischen und analytischen Lösung
        ist für \(n \geqslant 5000\) kleiner als \(10^{-3}\).

        Auch Abbildung 5 zeigt dies nochmal durch einen Plot des Fehlers an den verschiedenen Diskretisierungspunkten.
        Während der Fehler zwar variiert, so wird er für größere Werte von \(n\) immer kleiner und liegt beispielsweise
        für \(n = 128\) nur noch ungefähr 0.00005 und 0.00030. 
\end{itemize}


\section{Zusammenfassung}



\printbibliography
\end{document}
