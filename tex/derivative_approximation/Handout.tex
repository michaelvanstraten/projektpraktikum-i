\documentclass{scrartcl}
\usepackage{scrhack}  % resolves deprecated warnings in context of float and KOMA

%%% GENERAL PACKAGES %%%%%%%%%%%%%%%%%%%%%%%%%%%%%%%%%%%%%%%%%%%%%%%%%%%%%%%%%%
% inputenc allows the usage of non-ascii characters in the LaTeX source code
\usepackage[utf8]{inputenc}


%%% KOMA OPTIONS %%%%%%%%%%%%%%%%%%%%%%%%%%%%%%%%%%%%%%%%%%%%%%%%%%%%%%%%%%%%%%
\KOMAoptions{
  % font size in pt
  fontsize=12,
  % parameter for size of type area, cf. KOMA script documentation
  DIV=13,
  % option for two-sided documents : false, semi, true
  twoside=false,
  % option for paragraph skip :
  %   false - indent, without vertical space between two paragraphs
  %   true  - vertical space between two paragraphs, without indent
  parskip=true,
  % print 'chapter' before the chapter number
  chapterprefix=true,
  % option to generate a draft version
  draft=false
}


%%% TITLEPAGE %%%%%%%%%%%%%%%%%%%%%%%%%%%%%%%%%%%%%%%%%%%%%%%%%%%%%%%%%%%%%%%%%
% general description of the type of the document
\subject{Bericht Handout}
% head line of the title page, here with department and logo
\titlehead{%
  \begin{minipage}{.7\textwidth}%
  Humboldt-Universit\"at zu Berlin\\
  Mathematisch-Naturwissenschaftliche Fakult\"at\\
  Institut f\"ur Mathematik
  \end{minipage}
}
% title of the document
\title{Fehleranalyse der Finiten Differenzen Methode zum Approximieren der ersten und zweiten Ableitung}
% optional subtitle
%\subtitle{Draft from~\today}
% information about the author
\author{%
  Eingereicht von Autor, M. van Straten und P. Merz
}
% date of submission
\date{\today}



%%% LANGUAGE %%%%%%%%%%%%%%%%%%%%%%%%%%%%%%%%%%%%%%%%%%%%%%%%%%%%%%%%%%%%%%%%%%
% babel provides hyphenation patterns and translations of keywords like 'table
% of contents'
\usepackage[ngerman]{babel}


%%% MATH %%%%%%%%%%%%%%%%%%%%%%%%%%%%%%%%%%%%%%%%%%%%%%%%%%%%%%%%%%%%%%%%%%%%%%
% amsmath provides commands for type-setting mathematical formulas
\usepackage{amsmath}
% amssymb provides additional symbols
\usepackage{amssymb}
% HINT
% Use http://detexify.kirelabs.org/classify.html to find unknown symbols!

\usepackage{cite}

%%% MATH ENVIRONMENTS %%%%%%%%%%%%%%%%%%%%%%%%%%%%%%%%%%%%%%%%%%%%%%%%%%%%%%%%%
% amsthm provides environments for typical math text structure
\usepackage{amsthm}
\theoremstyle{remark}
\newtheorem*{bemerkung}{Bemerkung}


%%% COLORS %%%%%%%%%%%%%%%%%%%%%%%%%%%%%%%%%%%%%%%%%%%%%%%%%%%%%%%%%%%%%%%%%%%%
% define own colors and use colored text
\usepackage[pdftex,svgnames,hyperref]{xcolor}


%%% FONT %%%%%%%%%%%%%%%%%%%%%%%%%%%%%%%%%%%%%%%%%%%%%%%%%%%%%%%%%%%%%%%%%%%%%%
% choose latin modern font (a good general purpose font)
\usepackage{lmodern}
% fontenc and microtype improve the appearance of the font
\usepackage[T1]{fontenc}
\usepackage{microtype}


%%% HYPERLINKS %%%%%%%%%%%%%%%%%%%%%%%%%%%%%%%%%%%%%%%%%%%%%%%%%%%%%%%%%%%%%%%%
% automatic generation of hyperlinks for references and URIs
\usepackage{hyperref}


%%% GRAPHICAL ELEMENTS %%%%%%%%%%%%%%%%%%%%%%%%%%%%%%%%%%%%%%%%%%%%%%%%%%%%%%%%
% provides commands to include graphics
\usepackage{graphicx}


%%% TABLES %%%%%%%%%%%%%%%%%%%%%%%%%%%%%%%%%%%%%%%%%%%%%%%%%%%%%%%%%%%%%%%%%%%%
% provides commands for good-looking tables
\usepackage{booktabs}

%%%% more fonts: e.g. fraktur
\usepackage{yfonts}

%%% Code Listings %%%%%%%%%%%%%%%%
% provides commands for including code (python, latex, ...)
\usepackage{listings}
\definecolor{keywords}{RGB}{255,0,90}
\definecolor{comments}{RGB}{0,0,113}
\definecolor{red}{RGB}{160,0,0}
\definecolor{green}{RGB}{0,150,0}
\lstset{language=Python, 
        basicstyle=\ttfamily\small, 
        keywordstyle=\color{keywords},
        commentstyle=\color{comments},
        stringstyle=\color{red},
        showstringspaces=false,
        identifierstyle=\color{green},
        }

%%%% Grafiken zeichnen mit tikz
\usepackage{tikz}
% Definitionen f\"ur das Flowchart
\usetikzlibrary{shapes.geometric, arrows, shapes.multipart}
\tikzstyle{startstop} = [ellipse, minimum width=2.5cm, minimum height=0.8cm, 
	align = left,text centered, draw=black, fill=red!30]
\tikzstyle{io} = [trapezium, trapezium left angle=70, trapezium right angle=110,
	minimum width=2.5cm, minimum height=0.8cm, align = left, text centered, 
	draw=black, fill=blue!30]
\tikzstyle{process} = [rectangle, minimum width=2.5cm, minimum height=0.8cm, 
	align = left, text centered, draw=black, fill=gray!30]
\tikzstyle{err} = [rectangle, minimum width=2.5cm, minimum height=0.8cm, 
	align = left, text centered, draw=black, fill=red!10,text width=3cm]
\tikzstyle{subrout} = [rectangle split,rectangle split horizontal,
	minimum width=2.5cm, minimum height=0.8cm,align = left, text centered, 
	draw=black, fill=orange!30,rectangle split parts=3]
\tikzstyle{decision} = [diamond,shape aspect=2.5, minimum width=2.5cm, 
	minimum height=0.8cm, align = left, text centered, draw=black, 
	fill=green!30]
\tikzstyle{arrow} = [thick,->,>=stealth]




%%% MY USER-DEFINED COMMANDS %%%%%%%%%%%%%%%%%%%%%%%%%%%%%%%%%%%%%%%%%%%%%%%%%%
% define your own commands and abbreviations with \newcommand
% or \DeclareMathOperator
% e.g. try \R and \Span in math mode
\newcommand\R{\mathbb R}
\DeclareMathOperator\Span{span}


%%% BEGIN OF DOCUMENT %%%%%%%%%%%%%%%%%%%%%%%%%%%%%%%%%%%%%%%%%%%%%%%%%%%%%%%%%
% the type-setting of the document starts here
\begin{document}

% generating the title page
\maketitle
% generating the table of contents (requires to run pdflatex twice!)
\tableofcontents
% start a new page
\cleardoublepage


%%% BEGIN OF CONTENT %%%%%%%%%%%%%%%%%%%%%%%%%%%%%%%%%%%%%%%%%%%%%%%%%%%%%%%%%%
\section{Einleitung und Motivation}
Finite Differenzen sind ein grundlegendes Werkzeug der numerischen Mathematik, denn sie bieten die 
Möglichkeit Ableitungen erster und höherer Ordnung durch diskrete Punkte zu approximieren. 
Diese Finiten Differenzen können benutzt werden um gewöhnliche, wie auch partielle Differentialgleichung 
numerisch zu lösen. Dadurch finden finite Differenzen Anwendung in vielen Fachbereichen wie zum Beispiel der 
Physik oder dem Ingenieurswesen. \cite{FiniteDifferenzen} 




%%TODO CITE PPI MOODLE COURSE FOR THEORETICAL PART
\section{Theorie}
Sei \(f\) auf einem Intervall \([a,b]\) unendlich oft differenzierbar. Dann gilt für \(x \in (a,b)\) und \(h > 0 \) mit \(x + h \in [a,b]\):
\begin{equation}
  f(x + h) = \sum_{n=0}^{\infty}\frac{f^{n}(x)}{n!}h^{n} = f(x) + f^{\prime}(x)h + \sum_{n=2}^{\infty}\frac{f^{n}(x)}{n!}h^{n}
\end{equation}
und damit
\[ D_{h, r}^{(1)}f(x) := \frac{f(x + h) - f(x)}{h} = f^{\prime}(x) + \sum_{n=2}^{\infty}\frac{f^{n}(x)}{n!}h^{n} = f^{\prime}(x) + \mathcal{O}(h)\]
bzw. für \(x - h \in [a,b]\)
\begin{equation}
  f(x - h) = \sum_{n=0}^{\infty}\frac{f^{n}(x)}{n!}(-h)^{n} = f(x) - f^{\prime}(x)h + \sum_{n=2}^{\infty}\frac{f^{n}(x)}{n!}(-h)^{n}
\end{equation}
folgt
\[ D_{h, l}^{(1)}f(x) := \frac{f(x) - f(x - h)}{h} = f^{\prime}(x) - \sum_{n=2}^{\infty}\frac{f^{n}(x)}{n!}(-h)^{n} = f^{\prime}(x) + \mathcal{O}(h)\]
Die Subtraktion von (2) von (1) führt zu
\[ D_{h, c}^{(1)}f(x) := \frac{f(x + h) - f(x - h)}{2h} = f^{\prime}(x) + \frac{f^{\prime\prime\prime}(x)}{6}h^{2} = f^{\prime}(x) + \mathcal{O}(h^{2})\]
und die Addition von (1) und (2) zu
\[ D_{h}^{(2)}f(x) := \frac{f(x + h) -2f(x) + f(x - h)}{h^2} = f^{\prime\prime}(x) + \frac{f^{(4)}(x)h^{2}}{3 \cdot 4} = f^{\prime}(x) + \mathcal{O}(h^{2})\]

\(D_{h, r}^{(1)}f(x), D_{h, l}^{(1)}f(x) \text{ und } D_{h, c}^{(1)}f(x)\) bezeichen die erste rechtsseitige, linksseitige und zentrale finite Differenz von \(f\).
\(D_{h}^{(2)}f(x)\) bezeichnet die zweite finite Differenz von \(f\).

Damit erhält man eine Näherung des Approximationsfehlers in der Maximumsnorm mit \(p \in \mathbb{N} \text{ und } x_{i} := a + i\cdot\frac{|(b - a)|}{p}\) durch
\[e_{f}^{k}(h) := \underset{i = 0, \dots p}{max}|(f^{(k)}(x_{i}) - D_{h}^{(k)}f(x_{i}))|\]

\section{Experimente}
Für unsere Experimente haben wir uns die Funktion \(f(x) = \frac{sin(x)}{x}\) auf dem Intervall \(I :=[\pi, 3\pi]\) angeschaut. Offensichtlich is \(f\) auf \(I\) differenzierbar, also
lassen sich die im Theorie Teil dargestellten Methoden hier nutzen.

Einerseits haben wir die Graphen der exakten ersten und zweiten Ableitung mit den Graphen der Approximationen für verschiedene Werte von \(h\) verglichen.

%%TODO ADD FIGURE

Was wir uns ebenfalls angeschaut haben, ist der Approximationsfehler \(e_{f}^{k}(h)\) für die verschiedenen finiten Differenz, indem wir den Fehler, für ein Intervall
was von sehr kleinen bis zu sehr großen Werten von \(h\) reicht, geplottet haben.

%%TODO ADD FIGURE




\section{Auswertung}



\section{Zusammenfassung}




\printbibliography
\end{document}
